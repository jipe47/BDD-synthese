\chapter{Les bases de données déductives}

L'idée est de coupler une bdd à un ensemble de règles logiques pour en déduire de l'information, et sans passer par un langage de programmation.

Une base de données déductive est constituée

\begin{itemize}
	\item de prédicats de base (ou extensionnels), dont l'extension est conservée explicitement dans la base de données (base de données extensionnelle)
	\item de prédicats dérivés (ou intentionnels), dont l'extension est définie par des règles déductives (base de données intentionnelle)
\end{itemize}


	\section{Datalog}
	
		\subsection{Syntaxe}
		
		Un terme est soit une variable, soit une constante. Une formule atomique ou un atome est un symbole prédicatif appliqué à une liste d'arguments qui sont soit des variables, soit des constantes.
		
		Un symbole prédicatif peut être
		
		\begin{itemize}
			\item un nom de prédicat de base (extensionnel) (ex : parent(X,Y))
			\item un prédicat dérivé (intentionnel) (ex : grandparent(X,Y))
			\item une relation arithmétique
		\end{itemize}
		
		Ex : $grandparent(X,Y) \leftarrow parent(X,Z) and parent(Z, Y)$
		
		Restrictions :
		
		\begin{itemize}
			\item les prédicats apparaissant dans la tête (membre de gauche) de règles sont uniquement des prédicats dérivés, les règles ne définissent donc que des prédicats dérivés.
			\item conditions de sûreté : toute variable utilisée dans une règle doit apparaître dans au moins un atome dont le prédicat représente une relation, et qui n'est pas précédé d'une négation.
			
			Cela permet de garantir que l'évaluation des règles est possible.
		\end{itemize}
		
		Ex :
			
		\begin{itemize}
			\item Non sûr : $q(X, Y ) \leftarrow p(X,Z) \text{ AND NOT } r(X, Y,Z) AND X \geq Y$
			\item Sûr : $q(X) \leftarrow p(X) \text{ AND } X \geq 0$
		\end{itemize}
		
		
		\pagebreak
		\subsection{Interprétation des règles datalog}
		
		Calcul de l'extension des prédicats intentionnels définis par des règles datalog :
		
		\begin{enumerate}
			\item si un prédicat intentionnel est la tête de plusieurs règles, son extension est l'union des extensions données par les différentes règles
			\item pour chaque règle, on considère tous les tuples des relations extensionnelles non précédées d'une négation figurant dans le corps de la règle.
			
			Pour chaque combinaison de tuples, si
			
			\begin{itemize}
				\item les tuples sont compatibles avec les occurrences des prédicats extensionnels (attributs pour lesquels une valeur constante est donnée ayant cette valeur)
				\item les prédicats relationnels niés et les prédicats arithmétiques sont satisfaits
			\end{itemize}
			
			alors on ajoute le tuple défini par la règle.
		\end{enumerate}
		
		La sureté garantit que l'on a ainsi pris en compte toutes les valeurs possibles.
		
		
		\subsection{Conversion de l'algèbre relationnelle}
		
		\begin{itemize}
		\item $r \union s : rus(\overline{t_1}) \leftarrow r(t1)$ et $rus(\overline{t_1}) \leftarrow s(\overline{t_1})$
		\item $r - s : rms(\overline{t_1}) \leftarrow r(\overline{t_1}) \text{ AND NOT } s(\overline{t_1})$
		\item $r \times s : rts(\overline{t_1}, \overline{t_2}) \leftarrow r(\overline{t_1}) \text{ AND } s(\overline{t_2})$
		\item $r \Join s : rjs(\overline{t_1} \union \overline{t_2}) \leftarrow r(\overline{t_1}) \text{ AND } s(\overline{t_2})$
		\item $\pi_Xr : pr(\overline{t_1} \intersection X) \leftarrow r(\overline{t_1})$
		\item $\sigma_Cr : sr(\overline{t_1}) \leftarrow r(\overline{t_1}) \text{ AND } C(\overline{t_1})$
		\end{itemize}
		
		\subsection{Structure des règles et récursivité}
		
		Pour un ensemble de règles donné, le graphe des règles est défini comme suit :
		
		\begin{itemize}
			\item il y a un noeud dans le graphe pour chaque symbole prédicatif
			\item il existe un arc d'un noeud $p$ à un noeud $q$ ssi il existe une règle de la forme $q(\dots) \leftarrow \dots p (\dots ) \dots$
		\end{itemize}
		
		Un ensemble de règles est récursif si son graphe comporte un cycle.
		
		\dessin{7}
		
		Pour évaluer des règles récursives, le principe est de ne mettre dans l'extension des prédicats intentionels récursifs que les tuples qui doivent y figurer.
		
		On commence par une extension vide pour ces prédicats, puis on applique les règles jusqu'à ce qu'on obtienne une extension stable, ce qu'on appelle un point fixe (fixpoint).
		
		On peut utiliser cette approche tant que les prédicats définis récursivement ne sont pas dans la portée d'un \texttt{not}.
		
		\pagebreak
		Exemple pour le prédicat ancêtre.
		
		\dessin{8}
		
		On peut utiliser la négation et la récursion en même temps, pour autant qu'elles n'interagissent pas. On impose une restriction : les règles sont stratifiées.
		
		Si une règle comporte une négation $q(\dots) \leftarrow \dots \text{ NOT } p(\dots) \dots$, il n'y a pas de chemin de $q$ à $p$ dans le graphe des règles.
		
		Le graphe des règles peut alors être stratifié, divisé en couches : il n'y a pas de récursion entre couches et dans une règle définissant un prédicat d'une couche, la négation n'est appliquée qu'à des prédicats des couches inférieures.
		
		On peut alors évaluer l'extension des prédicats dérivés par couche.
		
		\subsection{Expressivité}
		
		Les prédicats définis par des règles non récursives peuvent être exprimés en algèbre relationnelle.
		
		Les prédicats définis par des règles récursives ne sont pas toujours exprimables. Par exemple, la fermeture transitive d'une relation (comme la relation ancetre) : son calcul nécessite l'application d'un nombre d'opérations qui dépend du contenu de la base de données.