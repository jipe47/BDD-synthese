\chapter{Gestion des transactions}

Pour maintenir la cohérence d'une base de données, on utilise les transactions : ce sont des suites d'opérations (lectures-écritures) effectuées sur une base de données.

Le concept s'utilise comme suit :

\begin{enumerate}
	\item l'interaction avec la BDD se fait uniquement avec des transactions
	\item le système de BDD gère les transactions de manière atomique ; si une transaction est exécutée, elle doit l'être entièrement et de façon indivisible en cas de parallélisme.
\end{enumerate}

	On peut définir les transactions dites ACID : 
	
	\begin{itemize}
		\item Atomicity : l'exécution des transactions doit être atomique
		\item Consistency : l'exécution des transactions doit respecter les contraintes d'intégrité définies par la BDD
		\item Isolation : il n'y a pas d'interaction entre transactions exécutées en parallèle
		\item Durability : une fois la transaction effectuée, le résultat ne peut être perdu
	\end{itemize}
	
	\section{Ordonnancement}
	
	On définit un ordonnancement (schedule) d'un ensemble de transactions $\ens{T_1, \dots , T_k}$ est un ordre d'exécution des opérations $a_i^j$ des transactions $T_i$.
	
	
	Un ordonnancement est séquentiel (serial) lorsqu'il existe une permutation $\pi$ de $\ens{1 , \dots , k}$ telle que l'ordonnancement est $T_{\pi(1)}, T_{\pi(2)}, \dots , T_{\pi(k)}$
	
	Un ordonnancement est séquentialisable (serializable) s'il existe un ordonnancement de $\ens{T_1, \dots , T_k}$ qui "donne le même résultat".
	
	Il faut un critère précis pour déterminer quand un ordonnancement est séquentialisable et un protocole d'exécution des transactions pour garantir la séquentialisabilité.


		\subsection{Critère de séquentialisabilité basé sur les conflits}
		
		Deux actions $a_1$ et $a_2$ sont en conflit si l'exécution de $a_1;a_2$ peut donner un résultat différent de $a_2;a_1$.
		
		\paragraph{Critère} Un ordonnancement est séquentialisable par rapport aux conflits s'il peut être transformé en un ordonnancement séquentiel par permutations successives d'actions qui ne sont pas en conflit.
		
		Un ordonnancement peut être séquentialisable sans l'être par rapport aux conflits, l'intérêt du critère "par rapport aux conflits" est qu'il est plus facile à exploiter.
		
		\paragraph{Déterminer la séquentialisabilité} Une transaction $T_i$ précède une transaction $T_j$ dans un ordonnancement donné s'il existe une action $a_i$ de $T_i$ qui précède une action $a_j$ de $T_j$ avec laquelle elle est en conflit.
		
		Le graphe de précédence d'un ordonnancement d'un ensemble de transactions $T = \ens{T_1, \dots , T_k}$ est le graphe $(V, E)$ tel que
		
		\begin{itemize}
			\item l'ensemble des sommets $V$ est l'ensemble des transactions $T$
			\item l'ensemble des arcs E est l'ensemble des paires $(T_i, T_j)$ telles que $T_i$ précède $T_j$ dans l'ordonnancement donné
		\end{itemize}
		
		\underline{Théorème} : un ordonnancement est séquentialisable par rapport aux conflits ssi son graphe de précédence est sans cycle.
		
		\underline{Preuve}
		
		\begin{itemize}
			\item[$\Leftarrow$] si le graphe est sans cycle, il contient au moins un noeud dans lequel n'entre aucun arc.
			
			Les actions de la transaction correspondant à ce noeud ne sont donc en conflit avec aucune action qui les précède dans l'ordonnancement.
			
			Les actions peuvent être donc déplacées vers le début de l'ordonnancement en ne les permutant qu'avec des actions avec lesquelles elles ne sont pas en conflit
			
			En répétant le même raisonnement avec le graphe dont le noeud traité a été éliminé, on peut construire de proche en proche un ordonnancement séquentiel correspondant à l'ordonnancement de départ
			
			\item[$\Rightarrow$] Supposons donc que dans le graphe de précédence, il y a un cycle $T_1 \rightarrow T_2 \rightarrow \dots \rightarrow T_n \rightarrow T_1$.
			
			Vu que la permutation d'actions qui ne sont pas en conflit ne change par le graphe de précédence, dans un ordonnancement séquentiel correspondant, les actions de $T_1$ doivent précéder celles de $T_2$,\dots qui précèdent celles de $T_n$, qui précèdent celles de $T_1$, ce qui est impossible.
			\end{itemize}
			
	\section{Algorithme d'ordonnancement : les verrous}
	
	Pour garantir la séquentialisabilité, on peut utiliser l'exclusion mutuelle des transactions (par exemple avec un sémaphore), mais cette contrainte est souvent trop restrictive.
	
	On va supposer que l'on n'assure l'exclusion mutuelle qu'au niveau des lectures/écritures. On utilise deux opérations :
	
	\begin{itemize}
		\item $lock(D, mode)$ ou $mode$ est une lecture $r$ où une écriture $w$
		\item $unlock(D)$
	\end{itemize}
		
	La règle est de verrouiller une donnée lors de sa première utilisation et de la déverrouiller après sa dernière utilisation.
	
		\subsection{Règle des 2 phases (2 phase locking)}
		
		Chaque donnée est verrouillée avant sa première utilisation et déverrouillée après sa dernière utilisation.
		
		Aucune opération de verrouillage ne peut suivre une opération de déverrouillage de la même transaction. On distingue donc deux phases : verrouillage et déverrouillage.
		
		\underline{Théorème} : si un ordonnancement est obtenu en respectant la règle des deux phases, il est séquentialisable.
		\pagebreak
		
		\underline{Preuve}
		
		\dessin{19}
		
		Le problème est que la règle des deux phases garantit la séquentialisabilité, mais elle n'exclut pas la présence de blocages (deadlocks) ou d'exclusion d'un processus (starvation).
		
		\dessin{20}
		
		
		\subsection{Prévention des blocages}
		
		On impose un ordre sur les transactions : $T_i < T_j$ si $T_i$ a démarré avant $T_j$.
		
		On applique la politique suivante : si une transaction $T_i$ dit avoir accès à une donnée verrouillée par $T_j$, alors
		
		\begin{itemize}
			\item si $T_i > T_j$, $T_i$ attend
			\item si $T_i < T_j$, la transaction $T_j$ est annulée (rolled back) et $T_i$ poursuit son exécution
		\end{itemize}
		
		La transaction la plus ancienne poursuit toujours son opération et il n'y a pas de blocage possible.
		
		\subsection{Récupération en cas d'erreurs}		
		
		 Le problème est de gérer les transactions qui ont été interrompues.
		
		Les erreurs possibles :
		
		\begin{itemize}
			\item annulstion d'une transaction
			\item arrêt du système
			\item perte d'une partie du contenu du disque
		\end{itemize}
		
		\pagebreak
		Modèle utilisé :
		
		\begin{itemize}
			\item les données sont transférées de et vers le disque bloc par bloc (page par page) par des opérations atomiques
			\item chaque transaction $T$ se termine par
			
			\begin{itemize}
				\item soit un $commit(T)$ : les modifications sont appliquées à la BDD
				\item soit un $abort(T)$ : la transaction est annulée, les modifications ne sont pas appliquées
			\end{itemize}
		\end{itemize}
		
		\subsection{Fichier historique}
		
		On utilise un fichier log pour pouvoir récupérer des erreurs ; on y sauve
		
		\begin{itemize}
			\item le début de chaque transaction
			\item chaque modification effectuée par une transaction
			\item les opérations $commit(T)$ effectuées
			\item les points de synchronisation
		\end{itemize}
		
		Lorsqu'une transaction modifie une page, cela est directement logé, mais les pages ne sont pas modifiées (on retarde donc les opérations unlock).
		
		Si la transaction se termine normalement, $commit(T)$ est ajouté au log, qui est sauvé sur le disque.
		
		Les modifications effectuées par la transaction sont effectivement réalisées avec unlock, les pages correspondantes sont sauvées sur le disque ou non.
		
		On impose périodiquement un point de synchronisation (checkpoint) : les nouvelles transactions sont interdites et on attend que toutes les transactions en cours soient terminées. On sauve ensuite toutes les pages modifiées sur le disque.
		
		Méthodes de récupération :
		
		\begin{itemize}
			\item arrêt d'une transaction : ne rien faire, car les modifications ne sont pas effectuées
			\item arrêt du système : avec le fichier historique, on réexécute tous les transactions dont le commit se trouve après le dernier point de synchronisation
			\item perte de contenu, idem, mais à partir du dernier point de synchronisation correspondant à une sauvegarde que l'on peut restaurer
		\end{itemize}
		
		Le log ne peut être perdu, c'est pour ça qu'il est dupliqué.
		
	\section{Transactions en SQL}
	
	Il y a une gestion complète des transactions ; la fin d'une transaction est signalée par la commande \texttt{commmit} ou \texttt{rollback}. Le début est signalé par \texttt{start transaction}.
	
	Il y a également un verrouillage de tables et un sauvetage immédiat sur disque des modifications. Les verrouillage et déverrouillage se font avec \texttt{lock tables} et \texttt{unlock tables}.
	
	Il y a plusieurs modes de gestion des transaction : \texttt{set transaction isolation level \dots}
	
	\begin{itemize}
		\item \texttt{serializable} : le plus contraignant, les transactions sont effectuées de façon séquentialisables
		\item \texttt{read uncommited} : la lecture de données modifiées par d'autres actions qui ne sont pas achevées (commit exécuté) est possible
		\item \texttt{read commited} : on ne lit que des données modifiées par des transactions terminées, mais des lectures multiples des mêmes données peuvent reproduire un résultat différent
		\item \texttt{reapatable read} : on ne voit que des données modifiées par des transactions terminées et les lectures multiples donnent le même résultat, si ce n'est que des nouveaux tuples peuvent apparaître entre deux lectures.
	\end{itemize}
	
	Le choix de niveau d'isolation est un compromis entre fiabilité, facilité d'utilisation et performances. La définition exacte des niveaux d'isolation dépend du système utilisé.